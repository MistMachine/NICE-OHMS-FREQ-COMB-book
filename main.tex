%%%%%%%%%%%%%%%%%%%%%%%%%%%%%%%%%%%%%%%%%%%%%%%%%%%%%%%%%%%

%% document class
\documentclass[a4paper]{book}

%% packages
% !TeX root = main
%% packages
\usepackage[utf8]{inputenc}
\usepackage{csvsimple}
\usepackage{float}

\usepackage{mathtools}					%for math
\usepackage{setspace}
\usepackage{hyperref}

\usepackage{graphicx}					%for importing images
\usepackage{wrapfig}

\usepackage[dvipsnames]{xcolor}
%\usepackage[margin=2cm]{geometry}
\usepackage{titlesec} %for formating headings%
\usepackage{caption} %captions placement

%% for hyperlinks
\hypersetup{
	colorlinks=true,
	linkcolor=blue,
	filecolor=magenta,      
	urlcolor=cyan,
}

\setcounter{secnumdepth}{4}

\makeindex

%% page settings
% !TeX root = main

%% page settings
\usepackage[utf8]{inputenc}
\usepackage[top=2cm, bottom=1.8cm,left=2.5cm,right=2.5cm]{geometry} % needed for page border settings
%%%%%%%%%%%%%%%%%%%%%%%%%%%%%%%%%%%%%%%%%%%%%%%%%%%%%%%%%%%
%\newcommand{\imginput}[1]{\input{#1}} %this command is to prevent pdf_latex imgs from showing up in structure
\begin{document}

	
	\frontmatter
	% !TeX root = main
%%%%%%%%%%%%%%%%%%%%%%%%%%%%%%%%%%%%%%%%%
% Minimalist Book Title Page 
% LaTeX Template
% Version 1.0 (27/12/12)
%
% This template has been downloaded from:
% http://www.LaTeXTemplates.com
%
% Original author:
% Peter Wilson (herries.press@earthlink.net)
%
% License:
% CC BY-NC-SA 3.0 (http://creativecommons.org/licenses/by-nc-sa/3.0/)
% 
% Instructions for using this template:
% This title page compiles as is. If you wish to include this title page in 
% another document, you will need to copy everything before 
% \begin{document} into the preamble of your document. The title page is
% then included using \titleTH within your document.
%
%%%%%%%%%%%%%%%%%%%%%%%%%%%%%%%%%%%%%%%%%

\begin{titlepage}
\newenvironment{bottompar}{\par\vspace*{\fill}}{\clearpage}

\raggedleft % Right-align all text
\vspace*{\baselineskip} % Whitespace at the top of the page

{\Large Johnson Anh Huy Nguyen}\\[0.167\textheight] % Author name

{\LARGE\bfseries A Complete Beginners Introduction To}\\[\baselineskip] % First part of the title, if it is unimportant consider making the font size smaller to accentuate the main title

{\textcolor{Red}{\Huge Cavity Enhanced Absorption Spectroscopy}}\\[\baselineskip] % Main title which draws the focus of the reader

{\Large \textit{Everything You Need to Know About Cavities, Lasers, Spectroscopy and Engineering Techniques For This Field}}\par % Tagline or further description

\vfill % Whitespace between the title block and the publisher

\vspace*{3\baselineskip} % Whitespace at the bottom of the page
\flushleft
This is just a place holder title page. This book is aimed at teaching people who are completely new to the field of {\bfseries{LASER SPECTROSCOPY}}. I will therefore be steering away from unnecessarily pretentious language while trying to be professional. I will be cracking jokes and hip references though, so not too professional. My goal is to teach in a fun manner, not sound like a pretentious stick in the mud snob. What I care about are results, not appearances.
\end{titlepage}

	
	\chapter{Preface}

		Currently, much of the material being taught today in university and college class are extremely out of touch with what is relevant with today's current research, particularly in physics. This has made it difficult for students nearing the end of their studies to jump into todays research projects without getting completely lost. Particular in the physics and physical chemistry field, research projects have become more technologically and theoretically complicated and rigourous. Long gone are the days of throwing in material and hoping for the best, boring titrations or smoothening metal surfaces by hand, etc etc. This book is aimed at helping young university students jump into laser spectroscopy to aid in acquisition of the necessary the theoretical and technical skills. This book is just a band-aid solution however and will provide as a base for my long-term future goal of updating the material being taught in elementary school all the way to the university/college level. 
		
		Another result of todays outdated teaching material is the large gap in experience and knowledge between todays professors and todays students, particularly again in physics. This large gap has made it difficult for professors to effectively communicate complicated concepts and techniques to students as they have long forgotten what it is like to be young and naive. Much of today's papers are also targeted at todays professors completely allienating todays students which is not helping at all.
		
		An examples of very complicated yet critical technique for many laser cavity setups is the infamouse Pound Drever Hall technique. Many physics students get completely lost with this technique because knowledge of feedback control theory, modulation and heterodyning is absolutely essential but is not being taught to many students at any point during their studies or at the very least, being given as pre-reading. Professors have essentially progressed many fields, not just laser spectroscopy or cold molecules, significantly and thus actually do not know how to effectively help and guide students in learning complex experiments. This book should explain concepts not currently explained to students from their studies so that they may understand this complex field. This is by no means easy and there is a reason why the content taught today is outdated.
		
		Formerly being one of those poor lost students and now one of the leaders of the field, I have gone on to solve the cavity problem, create a new cavity aligning technique(used all over the world), and created a systematic method of aligning a laser cavity in the infrared red region resulting in the alignment of the worlds first 0.98cm (~1m) mid-infrared cavity. I did all this while still in my early 20s during my undergrade. This book currently serves somewhat as a personal journal of everthing I have learned so as to provide a source at picking at my brain to help communicate everything I needed to learn in order to understand this complex interdisciplinary field. I have decided to make this book free for life on my github(or something else) repository so that anyone can freely view this book. I also will make a jupyter notebook page to accompany this book for interactivity with this book as some concepts are just better explained with interactive plots etc etc and images. The jupyter notebook will also serve to help young chemists and physicists learn python as I will comment the crap out of everything.
		
		At some point, I also would like python (more broadly, programming) to be taught teenagers as programming has proven to be a valuable tool in ALL FIELDS. Due to Python's easy to use and learn syntax, vast open source resource library of packages,  generous funding of various groups and societies, it has proven to become a high class, flexible, powerful and universal tool in many fields. Personally, being a chemist, physicist and soon engineer, I will just show be showcasing its uses in these fields but of it course has applications in statistic, finance, mathematics, and data science in general etc etc. I am aware of Julia but its library is almost non-existent.
		
		I still have much to learn, so this book(formerly a thesis) will continue to grow as I work on my PhD. A lot of this content will also end up in my more PhD thesis that I will hand in. This final thesis will be a professional document(for the snobs). I admit, I have torrented and never financialy contributed anything during my studies, so this will serve as my way of giving back to the academic community. I hope you old farts are ready. I also will not be posting experimental results like you would see in a typical thesis (hence why I call this a book and a seperate entity)

		I will clean up the language when things are finalized, I still got 6 more years. Made on Sunday, December 4, 2016 on Ubuntu 16.10 from my parents house (yeah I'm a currently a bum)
		
		\mainmatter
		
	\chapter{Introduction to Laser Spectroscopy}
	
\end{document}

